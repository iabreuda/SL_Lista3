\documentclass[a4paper, 12pt]{article}

\usepackage[portuges]{babel}
\usepackage[utf8]{inputenc}
\usepackage{amsmath}
\usepackage{indentfirst}
\usepackage{blindtext}
\usepackage{graphicx}
\usepackage[hidelinks]{hyperref}
\usepackage{gensymb}
\usepackage{pgfplots}

\author{Igor Abreu da Silva}

\title{Trabalho Final de Sistemas Lineares I}

\begin{document}

    \begin{titlepage}
        \begin{center}
            \huge{Universidade Federal do Rio de Janeiro}
            \vspace{95pt}

            \large{Lista IV - Sistemas Lineares I}
            \vspace{160pt}
        \end{center}

        \begin{flushleft}
            \begin{tabbing}
                Alunos\qquad\qquad\= Igor Abreu da Silva\\
                DRE\> 112053874 \\
                Curso\> Engenharia Eletrônica \\
                Turma\> 2016/2 \\
                Professor\> Natanael Nunes de Moura Junior \\

            \end{tabbing}

        \end{flushleft}

        \begin{center}
            \vspace{\fill}
            Rio de Janeiro, 16 de Novembro de 2016
        \end{center}
    \end{titlepage}

    \newpage
    \tableofcontents
    \listoffigures
    \thispagestyle{empty}
    \newpage
    \pagenumbering{arabic}

\section{Diagrama de P\'{o}los e Zeros}
    \subsection{Quest\~{a}o 1}
        \subsubsection{Item a}
        \subsubsection{Item b}
        \subsubsection{Item c}
\section{Propriedade da Transformada de Laplace}
    \subsection{Quest\~{a}o 2}
        \subsubsection{Item a}
        \subsubsection{Item b}
        \subsubsection{Item c}
    \subsection{Quest\~{a}o 3}
\section{Resposta em Frequ\^{e}ncia}
    \subsection{Quest\~{a}o 4}
        \subsubsection{Item a}
        \subsubsection{Item b}
        \subsubsection{Item c}
    \subsection{Quest\~{a}o 5}
        \subsubsection{Item a}
        \subsubsection{Item b}
        \subsubsection{Item c}
        \subsubsection{Item d}
        \subsubsection{Item e}
    \subsection{Quest\~{a}o 6}
        \subsubsection{Item a}
        \subsubsection{Item b}
        \subsubsection{Item c}
\section{Diagrama de Bode}
    \subsection{Quest\~{a}o 7}
        \subsubsection{Item a}
        \subsubsection{Item b}
        \subsubsection{Item c}
        \subsubsection{Item d}
    \subsection{Quest\~{a}o 8}
        \subsubsection{Item a}
        \subsubsection{Item b}
        \subsubsection{Item c}
        \subsubsection{Item d}
\end{document}